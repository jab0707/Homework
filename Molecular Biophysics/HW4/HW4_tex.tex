%%%%%%%%%%%%%%%%%%%%%%%%%%%%%%%%%%%%%%%%%%%%%%%%%%%%%%%%%%%%%%%%%%%%%%
\documentclass[12pt]{article}
\usepackage{setspace}
\usepackage[numbers,round]{natbib}
\bibliographystyle{apalike}
\usepackage{changebar}
\usepackage{tabularx}
\usepackage{graphicx}
\usepackage[margin=1in]{geometry}
\usepackage{comment}
\usepackage{matlab-prettifier}
%\usepackage{siunitx}
%\usepackage{seqsplit}
\textheight 9in
\textwidth 6.5in
%\topmargin -1in
%\oddsidemargin -.2in


%%%%%%%%%%%%%%%%%%%%%%% New commands

% These are handy commands you can use to create consistent and correct
% special characters, units, and expressions.
\newcommand{\etal}{{\em et al.}}
\newcommand{\etc}{{\em etc.}}
\newcommand{\eg}{{\em e.g.,}}
\newcommand{\ie}{{\em i.e.,}}
\newcommand{\cpp}{C{\raisebox{0.5ex}{\tiny++}}}
\newcommand{\degrees}{{$^{\circ}$}}
\newcommand{\muv}{${\rm \mu V}$}
\newcommand{\ohm}{$\Omega$}
\newcommand{\sft}{${\rm ft^2}$}


\begin{document}
	
	


\title{HW4: Molecular Biophysics}
\author{Jake Bergquist, u6010393 }
\maketitle

\section{Q1}
This is a simple case of C1V1 = C2V2. Here we have two starting concentrations ($C_1^o = 90 mM$ glucose and $C_2^o = 10 mM$ NaCl), and two ``volumes" (in this case the position of the membrane, X, is proportional to volume thus we can use that, and we will denote the two as $X_1^o=2$ and $X_2^o = 8$) We also know that at all times the total volume is 10 ($X_1^o + X_2^o = 0$). We can set up a simple system of equations as follows to solve for the final concentrations ($C_1^f$ and $C_2^f$) and ``volumes" ($X_1^f$, and $X_2^f$). We just also need to know that at equlibrium the concentrations of solutes on boths sides will equalize due to osmosis such that $C_1^f = C_2^f$\\
$
C_1^oX_1^o = C_1^fX_1^f\hspace{1in}(1)\\
C_2^oX_2^o = C_2^fX_2^f\hspace{1in}(2)\\
X_1^o + X_2^o = X_1^f + X_2^f = 10\hspace{0.151in}(3)\\
C_1^f = C_2^f\hspace{1.4in}(4)\\
$
First lets use these to solve for $X_2^f$\\
$
\frac{C_1^oX_1^o}{C_2^oX_2^o} = \frac{C_1^fX_1^f}{C_2^fX_2^f}\\
$
Apply $(4)$\\
$
\frac{C_1^oX_1^o}{C_2^oX_2^o} = \frac{C_2^fX_1^f}{C_2^fX_2^f} = \frac{X_1^f}{X_2^f}\\
X_2^f\frac{C_1^oX_1^o}{C_2^oX_2^o} = X_1^f\\
$
Apply $(3)$\\
$
X_2^f\frac{C_1^oX_1^o}{C_2^oX_2^o} = 10-X_2^f\\
X_2^f(\frac{C_1^oX_1^o}{C_2^oX_2^o}+1) = 10\\
X_2^f = \frac{10}{\frac{C_1^oX_1^o}{C_2^oX_2^o}+1} = \frac{10}{\frac{90*2}{10*8}+1} = 3.08 cm
$
Now that we have $X_2^f$ we can plug into $(2)$ to get $C_2^f = \frac{C_2^oX_2^o}{X_2^f} = 26 mM$ which is equal to $C_1^f$ which we can used to solve for $X_1^f = \frac{C_1^oX_1^o}{C_1^f} = 6.92 cm$.\\

Now all of this math is done assuming that $C_2^o$ is 10 mM, however because it id a salt it splits into two parts, thus the functional concentration of $C_2^o$ is 20 mM. Using this number we get:\\
$
C_1^f =C_2^f = 34 mM\\
X_1^f = 5.29 cm\\
X_2^f = 4.71 cm\\
$
I present both solutions because the former was advised during the homework assistance session but the latter follows the class lectures.

\subsection{Q2}
a) For this part we know that the change in number of subunits can be defined as $\frac{dN}{dt} = k_1[M] - K_{-1}$ where $k_1$ is the on rate and $K_{-1}$ is the off rate.Now if we integrate to get $\Delta N$ and $\Delta t$ we can solve for the time ($\Delta t$) it takes to bind the necessary number of subunits. If each subunit is 0.6 nm, then to make it to the edge of our 5000 nm distance we need 8334 units to bind, thus $\Delta N = 8334$. With this in mind $\Delta t = \frac{\Delta N}{k_1[m] - k_{-1}} = \frac{8334}{8.9*10 - 44} = 185 seconds$.

b) For this part we use a similar method as above but only concerning ourselves with the catastrophic off rates. We can either look at just the plus end or the plus and minus. Lets do both. Here $\Delta N$ is the same magnitude but opposite sign as in a because we need to lose that many for the filament to be broken down completely. \\
Both: $\Delta t = \frac{\Delta N}{-k_{-1}^+ - k_{-1}^-} = \frac{-8334}{-733 -915} = 5.05 sec$\\
Only plus: $\Delta t = \frac{\Delta N}{-k_{-1}^+} = \frac{-8334}{-733 } =  11.4 sec$\\

\subsection{Q3}
We have:\\
$D = \frac{K_bT}{6\pi \eta r} = \frac{1.38*10^{-23} J/K * 298K}{6\pi 0.001 Pa*sec * 1*10^{-9}m} = 2.18*10^{-13} m^2/s\\
$Here  Boltzman's constnat cancels mostly with temperature (298 K), Pascals per second (Viscocity) leaving meters squared per second.
We then plug that value fo $D$ in to :\\
$t = \frac{\delta^2}{2D} = \frac{(2.75*10^{-9})^2}{2*2.18*10^{-13} m^2/s} = 17 \mu s$\\
Now if the reaction is reaction limited then $\frac{1}{k_{on}C}<t$\\
Plugging in our values we find that the concentration needs to be at least $\frac{1}{k_{on}C} = t => \frac{1}{k_{on}t}=C = \frac{1}{0.000017 s * 12 \mu M^{-1}s^{-1}} = 4901 \mu M$\\

This is a very high concentration and likely means that in a cell the process is diffusion limited rather than reaction limited. That means that as soon as reactants become available they are used and the reaction can be controlled by controlling the availability of the reactants. Thus if it needs to speed up it can easily by flooding with more reactants. This method is easier to control in a biological system because the reaction rates are usually not easy to change.

\subsection{Q4}
Kinesin is a motor protein made up of two binding domains that bind to cytoskeletal elements, and a linker domain that link the two binding domains as well as link the motor protein to whatever substrate or vessicve it is pulling along with it as it moves. Kinesin is processive because of the ``hand over hand" method by which it moves. At all times there is at least one binding domain bound, and the other walks forward and binds. This allows for near continuous binding and unbinding cycles in which the motor proteins never dissociate from the filament.  Myosin II on the other hand binds, rachetws, upbinds, relaxes, and rebinds. Thus there is a period where the single binding domain is unbound, leading to a potential for complete dissopciation between filament and motor. This hpwever can be overcome by supporting protein structure that keep the filaments and motors lose to each other. This however makes the myosin motor non-processive and the movement is not continuous walking down the filament but rather more of pul, relax, pull, relax, as the binding domain shifts up the filament. Myosin has two binding heads but they work independently, while kinesin's two binding heads work together. Also myosin binds to actin while kinesin binds to tubulin. In myosin, binding of ATP causes a weakening of the actin binding, causing release fromt he filament fo the binding head. The ATP is hydrolized then causing the head region to advance to the next location. Docking the head group tot he new bind site causes the release of the phosphate group. This release allows to the energy stored from hydrolysis of ATP to drive the power stroke of the myosin head pulling on actin. At this point ADP is released, and ATP can bind restarting the cycle. During the recovery stroke where the myosin head unbinds and rebinds there is no binding to the filament. If there are many myosin motors then some of them will be bound when others are not thus there can be a certain number fo motors sucht that there is always one bound and moving. This number is very high for myosin II, meaning its duty ratio is low (amount of time spend unbound vs bound). However because for one single kinesin there is always at least one head bound (thus the duty ratios of the two heads must be at least 0.5), fewer are needed (onyl one actually) for continuous movement. Kinesin hydrolizes ATP in the lagging binding site as the leading site binds tightly to the tubulin filament.

\subsection{Q5}
We know that the product of all of the dissociation constants should be 1, thus we can solve for $K_0$ as $K_0 = \frac{1}{K_1K_2K_3K_4}.$ by definition\\
$
K_1 = \frac{10^{-4}}{2}\\
K_2 = \frac{10}{100}\\
K_3 = \frac{1}{0.1}\\
K_4 = \frac{0.5}{2}\\
K_0 = \frac{1}{(\frac{10^{-4}}{2})(\frac{10}{100})(\frac{1}{0.1})(\frac{0.5}{2})}\\
$
\end{document}








