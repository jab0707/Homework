\documentclass[12pt]{article}
\usepackage{graphicx}
\usepackage {color}
\usepackage{pdfpages}
\usepackage{float}
\usepackage{changebar}
\usepackage{enumitem,amssymb}
\renewcommand{\familydefault}{\sfdefault}
\usepackage[margin=1.2in]{geometry}
\usepackage{graphicx}
\usepackage{wrapfig}
\usepackage[super]{cite}
\usepackage{subcaption}
\usepackage[table]{xcolor}
\usepackage{amsmath}
\usepackage[sort, numbers]{natbib}
\usepackage{multirow}
\usepackage{tabularx}
\usepackage{siunitx}
\usepackage{seqsplit}
\usepackage{amsmath} % or simply amstext
\newcommand{\angstrom}{\text{\normalfont\AA}}
%%%%%%%%%%%%Defining the margins %%%%%%%%%%%%%%%%%%%%%
\textheight 9.in
\textwidth 6.5in
\topmargin -.5in
\oddsidemargin 0in
\setlength{\parskip}{\smallskipamount}

%%%%%%%%%%%%%%Specific Commands %%%%%%%%%%%%%%%%%%
\newcommand{\eg}{{\em e.g.,}}
\newcommand{\ie}{{\em i.e.,}}
\newcommand{\etc}{{\em etc.,}}
\newcommand{\etal}{{\em et al.}}
\newcommand{\degrees}{{$^{\circ}$}}
\newcommand{\fig}[1]{\textbf{Figure #1}}

%%%%%%%%%%%%%%%%%%%%%%%%%%%% Setting to control figure placement
% These determine the rules used to place floating objects like figures 
% They are only guides, but read the manual to see the effect of each.
\renewcommand{\topfraction}{.9}
\renewcommand{\bottomfraction}{.9}
\renewcommand{\textfraction}{.1}
\renewcommand{\familydefault}{\sfdefault} %setting the san serif font

%%%%%%%%%%%%%%%%%%%%%%%% Line spacing
% Use the following command for ``double'' spacing
%\setlength{\baselineskip}{1.2\baselineskip}
% and this one for an acceptable NIH spacing of 6lpi based on 11pt
%\setlength{\baselineskip}{.9\baselineskip}
% The baselineskip does not appear to work when we include a maketitle
% command in the main file.  Something there must set the line spacing
% If we use this next command, then things seem to work.
\renewcommand{\baselinestretch}{.9}

\setcounter{secnumdepth}{0} %make no numbers but have a table of contents


\begin{document}

\title{HW 1: Medical Imaging Systems}
\author{Jake Bergquist, u6010393 }
\maketitle

\section{Q1}\
\subsection{a}
I found the nitric oxide synthase heme domain at 1.65 \angstrom{} resolution in Bos taurus found via X-ray crystallography on rcsb.org. The catelog ID is 1D0C. The amino acid sequence of chain B is as follows:

\seqsplit{ SRAPAPATPHAPDHSPAPNSPTLTRPPEGPKFPRVKNWELGSITYDTLCAQSQQDGPCTPRRCLGSLVLPRKLQTRPSPG
	PPPAEQLLSQARDFINQYYSSIKRSGSQAHEERLQEVEAEVASTGTYHLRESELVFGAKQAWRNAPRCVGRIQWGKLQVF
	DARDCSSAQEMFTYICNHIKYATNRGNLRSAITVFPQRAPGRGDFRIWNSQLVRYAGYRQQDGSVRGDPANVEITELCIQ
	HGWTPGNGRFDVLPLLLQAPDEAPELFVLPPELVLEVPLEHPTLEWFAALGLRWYALPAVSNMLLEIGGLEFSAAPFSGW
	YMSTEIGTRNLCDPHRYNILEDVAVCMDLDTRTTSSLWKDKAAVEINLAVLHSFQLAKVTIVDHHAATVSFMKHLDNEQK
	ARGGCPADWAWIVPPISGSLTPVFHQEMVNYILSPAFRYQPDPW}

\subsection{b}
With the search criteria Thrombin, and homo sapiens I found that the thee most recent structures available are 6GBW, 6FJT, and 6EVV as of 9/3/19.

\section{Q2}
\subsection{a}
Position g in the structure has the highest proportion of amino acids that are charged at 4/5. The first of those is an arginine. It is at the amino terminal end which implies a pKa near 9.04. Even if this were a non terminal amino acid it would have a pKa of ~ 12.48. Thus this amino acid will be positively charged. The two are lysines which have a side chain pKa of 10.79, resulting in a positive charge. The next is a glutamic acid which has a side chain pKa of 4.25, resulting in a negative charge at neutral pH. The last amino acid in this position is a leucene which does not ionize as a side chain and thus will have a neutral charge.

\subsection{b}



%%%%%%%%%%%%%%%%%% Correct Bibliography Style




\end{document}








