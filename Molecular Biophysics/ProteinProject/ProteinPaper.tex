\documentclass[12pt]{article}
\usepackage{graphicx}
\usepackage {color}
\usepackage{pdfpages}
\usepackage{float}
\usepackage{changebar}
\usepackage{enumitem,amssymb}
\renewcommand{\familydefault}{\sfdefault}
\usepackage[margin=1.2in]{geometry}
\usepackage{graphicx}
\usepackage{wrapfig}
\usepackage[super]{cite}
\usepackage{subcaption}
\usepackage[table]{xcolor}
\usepackage{amsmath}
\usepackage[sort, numbers]{natbib}
\usepackage{multirow}
\usepackage{tabularx}
\usepackage{siunitx}

%%%%%%%%%%%%Defining the margins %%%%%%%%%%%%%%%%%%%%%
\textheight 9.in
\textwidth 6.5in
\topmargin -.5in
\oddsidemargin 0in
\setlength{\parskip}{\smallskipamount}

%%%%%%%%%%%%%%Specific Commands %%%%%%%%%%%%%%%%%%
\newcommand{\eg}{{\em e.g.,}}
\newcommand{\ie}{{\em i.e.,}}
\newcommand{\etc}{{\em etc.,}}
\newcommand{\etal}{{\em et al.}}
\newcommand{\degrees}{{$^{\circ}$}}
\newcommand{\fig}[1]{\textbf{Figure #1}}
\DeclareMathOperator*{\argmin}{argmin}
%%%%%%%%%%%%%%%%%%%%%%%%%%%% Setting to control figure placement
% These determine the rules used to place floating objects like figures 
% They are only guides, but read the manual to see the effect of each.
\renewcommand{\topfraction}{.9}
\renewcommand{\bottomfraction}{.9}
\renewcommand{\textfraction}{.1}
\renewcommand{\familydefault}{\sfdefault} %setting the san serif font

%%%%%%%%%%%%%%%%%%%%%%%% Line spacing
% Use the following command for ``double'' spacing
%\setlength{\baselineskip}{1.2\baselineskip}
% and this one for an acceptable NIH spacing of 6lpi based on 11pt
%\setlength{\baselineskip}{.9\baselineskip}
% The baselineskip does not appear to work when we include a maketitle
% command in the main file.  Something there must set the line spacing
% If we use this next command, then things seem to work.
\renewcommand{\baselinestretch}{.9}
\newcommand{\rpm}{\raisebox{.2ex}{$\scriptstyle\pm$}}
\setcounter{secnumdepth}{0} %make no numbers but have a table of contents


\begin{document}

\title{Troponin-C }
\author{Jake Bergquist, u6010393 }
\maketitle

\section{Introduction}
Troponin C is a protein found in skeletal and cardiac muscle that helps control the initiation of contraction of the muscle fiber. Troponin C makes up a regulatory complex of troponin proteins (troponin 1, C and L) which all act to modulate the binding of tropomyosin to actin filaments in the muscle. The function of the troponin complex, and by extension the function of troponin c, is critical to the proper timing, strength, and frequency of muscle contraction in both skeletal and cardiac muscle. Troponin is also used as a biochemical marker of heart health during potential instances of acute cardiac damage such as a myocardial infarction, as it is released into the blood when cardiomyocytes are damaged. Troponin C is a type of calcium binding protein with two distinct conformations that are stabilized by different concentrations of calcium. This allows troponin C to regulate muscle contraction (in conjunction with the rest of the troponin/tropomyosin complex) in a calcium dependent manner. Changes to troponin C are implicated in several disease processes, particularly of cardiac troponin C which plays a role in some forms of cardiomyopathy. Several pharmacological treatments of this disease state target troponin C.

\subsection{Structure and Function}
Troponin C is a 18 KDa  calcium binding protein of the EF-hand family of calcium binding proteins. 

\subsection{Homology and Binding}


\subsection{Isolation and Binding}


\section{Solution Studies}


\section{Interfacial Studies}


\section{Predictions}


\section{Calculations}


\section{Conclusion}


%\bibliography{library,biglit,strings}
%\bibliographystyle{IEEEtran}


\end{document}








