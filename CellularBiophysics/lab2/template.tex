\documentclass[11pt]{article}
\usepackage{graphicx}
\usepackage {color}
\usepackage{pdfpages}
\usepackage{float}
\usepackage{changebar}
\usepackage{enumitem,amssymb}
\renewcommand{\familydefault}{\sfdefault}
\usepackage[margin=1.2in]{geometry}
\usepackage{graphicx}
\usepackage{wrapfig}
\usepackage[super]{cite}
\usepackage{subcaption}
\usepackage[table]{xcolor}
\usepackage{amsmath}
\usepackage[sort, numbers]{natbib}

%%%%%%%%%%%%Defining the margins %%%%%%%%%%%%%%%%%%%%%
\textheight 9.in
\textwidth 6.5in
\topmargin -.5in
\oddsidemargin 0in
\setlength{\parskip}{\smallskipamount}

%%%%%%%%%%%%%%Specific Commands %%%%%%%%%%%%%%%%%%
\newcommand{\eg}{{\em e.g.,}}
\newcommand{\ie}{{\em i.e.,}}
\newcommand{\etc}{{\em etc.,}}
\newcommand{\etal}{{\em et al.}}
\newcommand{\degrees}{{$^{\circ}$}}

%%%%%%%%%%%%%%%%%%%%%%%%%%%% Setting to control figure placement
% These determine the rules used to place floating objects like figures 
% They are only guides, but read the manual to see the effect of each.
\renewcommand{\topfraction}{.9}
\renewcommand{\bottomfraction}{.9}
\renewcommand{\textfraction}{.1}
\renewcommand{\familydefault}{\sfdefault} %setting the san serif font

%%%%%%%%%%%%%%%%%%%%%%%% Line spacing
% Use the following command for ``double'' spacing
%\setlength{\baselineskip}{1.2\baselineskip}
% and this one for an acceptable NIH spacing of 6lpi based on 11pt
%\setlength{\baselineskip}{.9\baselineskip}
% The baselineskip does not appear to work when we include a maketitle
% command in the main file.  Something there must set the line spacing
% If we use this next command, then things seem to work.
\renewcommand{\baselinestretch}{.9}

\setcounter{secnumdepth}{0} %make no numbers but have a table of contents


\begin{document}

\title{Lab 2, Ion Channel Simulations}
\author{Jake Bergquist, u6010393}
\maketitle
\tableofcontents
\newpage

\section{Introduction}

\section{Methods}

\section{Results}

\rowcolors{2}{gray!25}{white}
\begin{table}[H]
	\centering
	\caption{Simulated measurements for conduction velocity and maximal upstroke velocity}
	\label{tab:results}
	\begin{tabular}{ccc}
		\hline \hline
		Experiment  & Conduction Velocity & Maximal Upstroke Velocity\\ 
		Number & (cm/ms)& (mV/ms) \\
		\hline
		1 & 16.6389 &  69.8933 \\ 
		2 &  18.9606&  73.9121 \\ 
		
		3 &  24.7062&  83.871 \\ 
		
		4&  45.2948&  109.1537\\ 
		
		5&  52.6004&  116.8785\\ 
		
		6&  77.6482&  131.6630\\ 
		
		7&  2.0641&  0.0222\\ 
		
		8&  44.0706&  109.3527\\ 
		
		9&  45.2948&  109.1675\\ 
		
		10&  2.1013&  -0.0024\\ 
		
		11&  2.0896&  -0.0024\\ 
		
		12&  60.3930&  179.3235\\ 
		13&  38.8214&  86.8933 \\ 
		
		14&  31.9728&  60.6956\\ 
		
		15&  27.6375&  48.7076\\ 
		16&  23.2945&  35.4873\\ 
		17&  20.3827&  28.0958\\ 
		\hline 
		\hline
	\end{tabular} 
\end{table}


\section{Discussion} 


%%%%%%%%%%%%%%%%%% Correct Bibliography Style
%\bibliography{/Users/Brian/GoogleDrive/Papers/library.bib}
%\bibliographystyle{IEEEtran}


\end{document}


%%%%%%%%%%%%%%%% Table Example %%%%%%%%%%%%%%%%%%%%%%
\rowcolors{2}{gray!25}{white}
\begin{table}[H]
	\centering
	\caption{Simulated measurements for conduction velocity and maximal upstroke velocity}
	\label{tab:results}
\begin{tabular}{ccc}
	\hline \hline
	Experiment  & Conduction Velocity & Maximal Upstroke Velocity\\ 
	Number & (cm/ms)& (mV/ms) \\
	\hline
	 1 & 16.6389 &  69.8933 \\ 
	 2 &  18.9606&  73.9121 \\ 

	 3 &  24.7062&  83.871 \\ 

	  4&  45.2948&  109.1537\\ 

	  5&  52.6004&  116.8785\\ 

	  6&  77.6482&  131.6630\\ 

	  7&  2.0641&  0.0222\\ 

	  8&  44.0706&  109.3527\\ 

	  9&  45.2948&  109.1675\\ 

	  10&  2.1013&  -0.0024\\ 

	  11&  2.0896&  -0.0024\\ 

	  12&  60.3930&  179.3235\\ 
	  13&  38.8214&  86.8933 \\ 

	  14&  31.9728&  60.6956\\ 

	  15&  27.6375&  48.7076\\ 
	  16&  23.2945&  35.4873\\ 
	  17&  20.3827&  28.0958\\ 
	\hline 
	\hline
\end{tabular} 
\end{table}

%%%%%%%%%%%%%%%%% Figure Example %%%%%%%%%%%%%%%%%%%
	\begin{figure}[H]
	\centering
	\begin{subfigure}{0.49\textwidth}
		\centering
		\includegraphics[width = \textwidth]{../Simulation/Experiment_11.png}
		\caption{}
		\label{fig:left}
	\end{subfigure}
	\begin{subfigure}{0.49\textwidth}
		\centering
		\includegraphics[width = \textwidth]{../Simulation/Experiment_9.png}
		\caption{}
		\label{fig:right}
	\end{subfigure}
	\vskip\baselineskip
	\begin{subfigure}{0.49\textwidth}
		\centering
		\includegraphics[width = \textwidth]{../Simulation/Experiment_4.png}
		\caption{}
		\label{fig:left}
	\end{subfigure}
	\begin{subfigure}{0.49\textwidth}
		\centering
		\includegraphics[width = \textwidth]{stimulation.png}
		\caption{}
		\label{fig:right}
	\end{subfigure}
	\caption{Changes in Stimulation Current. (a) Output figure from experiment 11. (b) Output figure from experiment 9. (c) Output figure from experiment 4. (d) A summary figure showing the changes in conduction velocity and max dV/dt as the stimulation current varies. Note that once you exceed a certain threshold there is relatively no change to the conduction velocity or max dV/dt. }
	\label{fig:stimulation}
\end{figure}




