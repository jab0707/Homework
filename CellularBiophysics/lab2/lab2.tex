\documentclass[11pt]{article}
\usepackage{graphicx}
\usepackage {color}
\usepackage{pdfpages}
\usepackage{float}
\usepackage{changebar}
\usepackage{enumitem,amssymb}
\renewcommand{\familydefault}{\sfdefault}
\usepackage[margin=1.2in]{geometry}
\usepackage{graphicx}
\usepackage{wrapfig}
\usepackage[super]{cite}
\usepackage{subcaption}
\usepackage[table]{xcolor}
\usepackage{amsmath}
\usepackage[sort, numbers]{natbib}

%%%%%%%%%%%%Defining the margins %%%%%%%%%%%%%%%%%%%%%
\textheight 9.in
\textwidth 6.5in
\topmargin -.5in
\oddsidemargin 0in
\setlength{\parskip}{\smallskipamount}

%%%%%%%%%%%%%%Specific Commands %%%%%%%%%%%%%%%%%%
\newcommand{\eg}{{\em e.g.,}}
\newcommand{\ie}{{\em i.e.,}}
\newcommand{\etc}{{\em etc.,}}
\newcommand{\etal}{{\em et al.}}
\newcommand{\degrees}{{$^{\circ}$}}
\newcommand{\fig}[1]{\textbf{Figure #1}}

%%%%%%%%%%%%%%%%%%%%%%%%%%%% Setting to control figure placement
% These determine the rules used to place floating objects like figures 
% They are only guides, but read the manual to see the effect of each.
\renewcommand{\topfraction}{.9}
\renewcommand{\bottomfraction}{.9}
\renewcommand{\textfraction}{.1}
\renewcommand{\familydefault}{\sfdefault} %setting the san serif font

%%%%%%%%%%%%%%%%%%%%%%%% Line spacing
% Use the following command for ``double'' spacing
%\setlength{\baselineskip}{1.2\baselineskip}
% and this one for an acceptable NIH spacing of 6lpi based on 11pt
%\setlength{\baselineskip}{.9\baselineskip}
% The baselineskip does not appear to work when we include a maketitle
% command in the main file.  Something there must set the line spacing
% If we use this next command, then things seem to work.
\renewcommand{\baselinestretch}{.9}

\setcounter{secnumdepth}{0} %make no numbers but have a table of contents


\begin{document}

\title{Lab 2, Ion Channel Simulations}
\author{Jake Bergquist, u6010393}
\maketitle
\tableofcontents
\newpage

\section{Introduction}
\par{}
While a larger goal of projects such as the Pysiome Project and the Virtual Physiological Human project are to develop tissue, organ, and system models for human Physiology, smaller scale models such as ion channel models are a vital part of these projects.\cite{Fink2011} Mathematical and computational models of Ion channels provide a framework for highly controlled investigation of the properties of these channels as well as the combination of vast amounts of experimental data into a single comprehensive model.  The modeling of ion channels contributes greatly to an effort to integrate and interpret experimental data as well as provide a highly detailed way to perform precise and otherwise technically impractical hypothesis testing on such a fine scale.
\par{}
While there are a vast variety of modeling techniques that range from basic current modeling with Hodgkin and Huxley to computationally intensive atomic scale molecular models, more frequently single ion channels are modeled using a Markovian chain of states model.\cite{Fink2011}\cite{Kojima2018a}
A Markovian model is one in which the next state of the model only depends on the current state, irrespective of the previous states. In the case of ion channels, Markovian models are typically structured into different closed and open states with rate constants that determine the probability of transitioning from one state to another. Each of these different states define different properties about the ion channel in the simulation, such as its permeability to ions, thus the effect on the conductance of those ions.\cite{Fink2011} In the most simple case of a two state Markovian ion channel model there is an open state and a closed state. At any given time the model may transition from closed to open or open to closed based on two rate constants, an open to closed rate, and a closed to open rate. If the closed to open rate constant were higher, this would describe a channel has a higher probability of being in the open state as opposed to the closed state. By increasing the number of states these models can be used to describe more complex behaviors such as channels with inactive states or drug/ligand binding that modulate activity. Each of these different state transitions have add rate coefficients describing the transitions between those states and other states, and thus a complex network can be built up.
\par{}
Each of the parameters that comprise the rate coefficients must be determined. Typically the values used come from direct experimental data from the channels of interest. This data includes structural information gained from crystallography, imaging data, genomic analysis, and perhaps most frequently from patch clamp electrical recordings. A patch clamp is a technique for assaying the electrical behavior of cells down to the single ion channel level. Either a specific cell expressing the channels of interest, such as a neuron or a cardiomyocyte, or a cell that has been made to express the channel of interest is isolated. A glass needle is used to either perforate or isolate a small section or patch of a cell membrane. In the case of perforation this allows access to the intracellular environment whereas in the case of isolation this allows access to the extracellular or intracellular membrane surface, depending on the preparation. In either case the concentrations of ions and other materials can be readily changed in the bath and within the needle. The voltage, injected current, or ion concentrations are then varied, and the resulting response of the ion channels can be measured via electrodes in the needle and extracellular space. Other parameters about the channels are derived from the structural information gained by crystallography and computational analysis given the amino acid sequence of the channels. These analyses are limited by the ability to isolate and crystallize the channels and the computation power and algorithms available to computer the atomic and amino acid scale interactions and models.
\par{}
As a result of years of work in this field on many different ion channels, vast amounts of data has been generated and funneled into constructing mathematical and computational models. The implementation of these models can often be difficult, however there are a variety of packages designed to incorporate the many different ion channel models into an centralized interface to allow for fast and robust testing of the models. Such libraries as CellML and applications like Jsim utilize these ionic models to allow researchers to perform robust simulations and tests with these models.
\par{}
In order to efficiently implement these various complex ion channel models, languages such as CellML have been developed. CellML is a markdown language built on top of XML as a way to standardize the development and sharing of models such as ion channel models. CellML allows researchers to easily share their models and parameters between different modeling environments, facilitating collaboration and distribution of developed models and findings. CellML itself is under active development and new improvements and tools are routinely implemented. Utilizing CellML as a standard language for writing models allows them to be easily shared and implemented in other settings. This significantly cuts down the time it takes to implement new models or test changes and additions to existing ones. Without a standard language like CellML the progress in developing and improving ion channel models would be hindered. For this lab we utilized Jsim as a simulation environment. Jsim is a java based simulation software developed by the Pysiome project for use in computational models and simulation based on experimental data.\cite{Fink2011} Jsim can read in CellML files and run the appropriate simulations, then display results in an interactive interface. Through Jsim, model parameters can be exposed for manipulation and testing, allowing for rapid and robust exploration of the model and effects of different parameters. Jsim allows for the testing of hypotheses and exploration of models in a lightweight, computationally inexpensive format that does not require significant experience or expertise with regards to developing CellML models.
\par{}
During this lab assignment we were required to simulate and manipulate membrane potential and ionic currents as well as a Markovian potassium channel model developed for a cardiac fibroblast.\cite{Sachse2008} Utilizing Jsim and the CellML cardiac fibroblast model developed by Dr. Frank Sachse we were able to explore the model and investigate the results from changing the different parameters such as ion conductances, stimulation protocols, and rate constants for the different Markovian states. Additionally the robust visualizations available through Jsim allowed for rapid visualization and interpretation of the model outputs and the effects of our changes and tests.

\section{Methods}
\subsection{1:Electrical Modeling of Membranes}
During this section we explored the relationship between the voltage at the membrane out our cardiac fibroblast and different stimulation protocols. Additionally we investigated the effects of ion currents on these membrane voltages.

\subsection{1.1}

For the first investigation we loaded the fibroblast CellML model described in \cite{Sachse2008} and configured it to be a purely capacitive system. To do so we set all of the ionic and background conductances to zero. Specifically those parameters were the background conductance (Gb) the potassium conductance (Gkir) and the potassium shaker conductance (Pshkr). Next we set the stimulation to begin at 0.1 seconds with a duration of 1 second, an amplitude of 0.2 nA and a frequency of 1 Hz. The simulation was then ran for 6 seconds and the membrane voltage was plotted as a function of time. The system we have made here is a capacitor-current source system. 

\subsection{1.2}
For the next section we took our purely capacitive cell from 1.1 and added a background (Gb) conductance of \ensuremath{1\times{}10^-3\mu S}. We also changed the reversal potential (Eb) to -84 mV. We then ran the same stimulation protocol as before with stimulation to begin at 0.1 seconds with a duration of 1 second, an amplitude of 0.2 nA and a frequency of 1 Hz. After 6 seconds of simulation we plotted membrane voltage, and membrane currents as a function of time. The system we created here is a resistor-capacitor system with a current source.

\subsection{2: Markovian Models of Ion Channels}
During these next sections we manipulated the parameters of the potassium shaker channel Markov model

\subsection{2.1}

\subsection{2.2}

\section{Results}


\section{Discussion} 


%%%%%%%%%%%%%%%%%% Correct Bibliography Style

\bibliography{C:/Users/Jake/Documents/library}
\bibliographystyle{IEEEtran}


\end{document}


%%%%%%%%%%%%%%%% Table Example %%%%%%%%%%%%%%%%%%%%%%
\rowcolors{2}{gray!25}{white}
\begin{table}[H]
	\centering
	\caption{Simulated measurements for conduction velocity and maximal upstroke velocity}
	\label{tab:results}
\begin{tabular}{ccc}
	\hline \hline
	Experiment  & Conduction Velocity & Maximal Upstroke Velocity\\ 
	Number & (cm/ms)& (mV/ms) \\
	\hline
	 1 & 16.6389 &  69.8933 \\ 
	 2 &  18.9606&  73.9121 \\ 

	 3 &  24.7062&  83.871 \\ 

	  4&  45.2948&  109.1537\\ 

	  5&  52.6004&  116.8785\\ 

	  6&  77.6482&  131.6630\\ 

	  7&  2.0641&  0.0222\\ 

	  8&  44.0706&  109.3527\\ 

	  9&  45.2948&  109.1675\\ 

	  10&  2.1013&  -0.0024\\ 

	  11&  2.0896&  -0.0024\\ 

	  12&  60.3930&  179.3235\\ 
	  13&  38.8214&  86.8933 \\ 

	  14&  31.9728&  60.6956\\ 

	  15&  27.6375&  48.7076\\ 
	  16&  23.2945&  35.4873\\ 
	  17&  20.3827&  28.0958\\ 
	\hline 
	\hline
\end{tabular} 
\end{table}

%%%%%%%%%%%%%%%%% Figure Example %%%%%%%%%%%%%%%%%%%
	\begin{figure}[H]
	\centering
	\begin{subfigure}{0.49\textwidth}
		\centering
		\includegraphics[width = \textwidth]{../Simulation/Experiment_11.png}
		\caption{}
		\label{fig:left}
	\end{subfigure}
	\begin{subfigure}{0.49\textwidth}
		\centering
		\includegraphics[width = \textwidth]{../Simulation/Experiment_9.png}
		\caption{}
		\label{fig:right}
	\end{subfigure}
	\vskip\baselineskip
	\begin{subfigure}{0.49\textwidth}
		\centering
		\includegraphics[width = \textwidth]{../Simulation/Experiment_4.png}
		\caption{}
		\label{fig:left}
	\end{subfigure}
	\begin{subfigure}{0.49\textwidth}
		\centering
		\includegraphics[width = \textwidth]{stimulation.png}
		\caption{}
		\label{fig:right}
	\end{subfigure}
	\caption{Changes in Stimulation Current. (a) Output figure from experiment 11. (b) Output figure from experiment 9. (c) Output figure from experiment 4. (d) A summary figure showing the changes in conduction velocity and max dV/dt as the stimulation current varies. Note that once you exceed a certain threshold there is relatively no change to the conduction velocity or max dV/dt. }
	\label{fig:stimulation}
\end{figure}




