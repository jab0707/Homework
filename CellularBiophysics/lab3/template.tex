\documentclass[11pt]{article}
\usepackage{graphicx}
\usepackage {color}
\usepackage{pdfpages}
\usepackage{float}
\usepackage{changebar}
\usepackage{enumitem,amssymb}
\renewcommand{\familydefault}{\sfdefault}
\usepackage[margin=1.2in]{geometry}
\usepackage{graphicx}
\usepackage{wrapfig}
\usepackage[super]{cite}
\usepackage{subcaption}
\usepackage[table]{xcolor}
\usepackage{amsmath}
\usepackage[sort, numbers]{natbib}

%%%%%%%%%%%%Defining the margins %%%%%%%%%%%%%%%%%%%%%
\textheight 9.in
\textwidth 6.5in
\topmargin -.5in
\oddsidemargin 0in
\setlength{\parskip}{\smallskipamount}

%%%%%%%%%%%%%%Specific Commands %%%%%%%%%%%%%%%%%%
\newcommand{\eg}{{\em e.g.,}}
\newcommand{\ie}{{\em i.e.,}}
\newcommand{\etc}{{\em etc.,}}
\newcommand{\etal}{{\em et al.}}
\newcommand{\degrees}{{$^{\circ}$}}
\newcommand{\fig}[1]{\textbf{Figure #1}}

%%%%%%%%%%%%%%%%%%%%%%%%%%%% Setting to control figure placement
% These determine the rules used to place floating objects like figures 
% They are only guides, but read the manual to see the effect of each.
\renewcommand{\topfraction}{.9}
\renewcommand{\bottomfraction}{.9}
\renewcommand{\textfraction}{.1}
\renewcommand{\familydefault}{\sfdefault} %setting the san serif font

%%%%%%%%%%%%%%%%%%%%%%%% Line spacing
% Use the following command for ``double'' spacing
%\setlength{\baselineskip}{1.2\baselineskip}
% and this one for an acceptable NIH spacing of 6lpi based on 11pt
%\setlength{\baselineskip}{.9\baselineskip}
% The baselineskip does not appear to work when we include a maketitle
% command in the main file.  Something there must set the line spacing
% If we use this next command, then things seem to work.
\renewcommand{\baselinestretch}{.9}

\setcounter{secnumdepth}{0} %make no numbers but have a table of contents


\begin{document}

\title{Lab 2, Ion Channel Simulations}
\author{Jake Bergquist, u6010393}
\maketitle
\tableofcontents
\newpage

\section{Introduction}
\par{}
While a larger goal of projects such as the Pysiome Project and the Virtual Physiological Human project are to develop tissue, organ, and system models for human Physiology, smaller scale models such as ion channel models are a vital part of these projects.\cite{Fink2011} Mathematical and computational models of Ion channels provide a framework for highly controlled investigation of the properties of these channels as well as the combination of vast amounts of experimental data into a single comprehensive model.  The modeling of ion channels contributes greatly to an effort to integrate and interpret experimental data as well as provide a highly detailed way to perform precise and otherwise technically impractical hypothesis testing on such a fine scale.
\par{}
While there are a vast variety of modeling techniques that range from basic current modeling with Hodgkin and Huxley to computationally intensive atomic scale molecular models, more frequently single ion channels are modeled using a Markovian chain of states model.\cite{Fink2011}\cite{Kojima2018a}
A Markovian model is one in which the next state of the model only depends on the current state, irrespective of the previous states. In the case of ion channels, Markovian models are typically structured into different closed and open states with rate constants that determine the probability of transitioning from one state to another. Each of these different states define different properties about the ion channel in the simulation, such as its permeability to ions, thus the effect on the conductance of those ions.\cite{Fink2011} In the most simple case of a two state Markovian ion channel model there is an open state and a closed state. At any given time the model may transition from closed to open or open to closed based on two rate constants, an open to closed rate, and a closed to open rate. If the closed to open rate constant were higher, this would describe a channel has a higher probability of being in the open state as opposed to the closed state. By increasing the number of states these models can be used to describe more complex behaviors such as channels with inactive states or drug/ligand binding that modulate activity. Each of these different state transitions have add rate coefficients describing the transitions between those states and other states, and thus a complex network can be built up.
\par{}
Each of the parameters that comprise the rate coefficients must be determined. Typically the values used come from direct experimental data from the channels of interest. This data includes structural information gained from crystallography, imaging data, genomic analysis, and perhaps most frequently from patch clamp electrical recordings. 

\section{Methods}
\subsection{1:Electrical Modeling of Membranes}

\subsection{1.1}

\subsection{1.2}


\subsection{2: Markovian Models of Ion Channels}
\subsection{2.1}

\subsection{2.2}

\section{Results}


\section{Discussion} 


%%%%%%%%%%%%%%%%%% Correct Bibliography Style

\bibliography{C:/Users/Jake/Documents/library}
\bibliographystyle{IEEEtran}


\end{document}


%%%%%%%%%%%%%%%% Table Example %%%%%%%%%%%%%%%%%%%%%%
\rowcolors{2}{gray!25}{white}
\begin{table}[H]
	\centering
	\caption{Simulated measurements for conduction velocity and maximal upstroke velocity}
	\label{tab:results}
\begin{tabular}{ccc}
	\hline \hline
	Experiment  & Conduction Velocity & Maximal Upstroke Velocity\\ 
	Number & (cm/ms)& (mV/ms) \\
	\hline
	 1 & 16.6389 &  69.8933 \\ 
	 2 &  18.9606&  73.9121 \\ 

	 3 &  24.7062&  83.871 \\ 

	  4&  45.2948&  109.1537\\ 

	  5&  52.6004&  116.8785\\ 

	  6&  77.6482&  131.6630\\ 

	  7&  2.0641&  0.0222\\ 

	  8&  44.0706&  109.3527\\ 

	  9&  45.2948&  109.1675\\ 

	  10&  2.1013&  -0.0024\\ 

	  11&  2.0896&  -0.0024\\ 

	  12&  60.3930&  179.3235\\ 
	  13&  38.8214&  86.8933 \\ 

	  14&  31.9728&  60.6956\\ 

	  15&  27.6375&  48.7076\\ 
	  16&  23.2945&  35.4873\\ 
	  17&  20.3827&  28.0958\\ 
	\hline 
	\hline
\end{tabular} 
\end{table}

%%%%%%%%%%%%%%%%% Figure Example %%%%%%%%%%%%%%%%%%%
	\begin{figure}[H]
	\centering
	\begin{subfigure}{0.49\textwidth}
		\centering
		\includegraphics[width = \textwidth]{../Simulation/Experiment_11.png}
		\caption{}
		\label{fig:left}
	\end{subfigure}
	\begin{subfigure}{0.49\textwidth}
		\centering
		\includegraphics[width = \textwidth]{../Simulation/Experiment_9.png}
		\caption{}
		\label{fig:right}
	\end{subfigure}
	\vskip\baselineskip
	\begin{subfigure}{0.49\textwidth}
		\centering
		\includegraphics[width = \textwidth]{../Simulation/Experiment_4.png}
		\caption{}
		\label{fig:left}
	\end{subfigure}
	\begin{subfigure}{0.49\textwidth}
		\centering
		\includegraphics[width = \textwidth]{stimulation.png}
		\caption{}
		\label{fig:right}
	\end{subfigure}
	\caption{Changes in Stimulation Current. (a) Output figure from experiment 11. (b) Output figure from experiment 9. (c) Output figure from experiment 4. (d) A summary figure showing the changes in conduction velocity and max dV/dt as the stimulation current varies. Note that once you exceed a certain threshold there is relatively no change to the conduction velocity or max dV/dt. }
	\label{fig:stimulation}
\end{figure}




