\documentclass[12pt]{article}
\usepackage{graphicx}
\usepackage {color}
\usepackage{pdfpages}
\usepackage{float}
\usepackage{changebar}
\usepackage{enumitem,amssymb}
\renewcommand{\familydefault}{\sfdefault}
\usepackage[margin=1.2in]{geometry}
\usepackage{graphicx}
\usepackage{wrapfig}
\usepackage[super]{cite}
\usepackage{subcaption}
\usepackage[table]{xcolor}
\usepackage{amsmath}
\usepackage[sort, numbers]{natbib}

%%%%%%%%%%%%Defining the margins %%%%%%%%%%%%%%%%%%%%%
\textheight 9.in
\textwidth 6.5in
\topmargin -.5in
\oddsidemargin 0in
\setlength{\parskip}{\smallskipamount}

%%%%%%%%%%%%%%Specific Commands %%%%%%%%%%%%%%%%%%
\newcommand{\eg}{{\em e.g.,}}
\newcommand{\ie}{{\em i.e.,}}
\newcommand{\etc}{{\em etc.,}}
\newcommand{\etal}{{\em et al.}}
\newcommand{\degrees}{{$^{\circ}$}}


%%%%%%%%%%%%%%%%%%%%%%%%%%%% Setting to control figure placement
% These determine the rules used to place floating objects like figures 
% They are only guides, but read the manual to see the effect of each.
\renewcommand{\topfraction}{.9}
\renewcommand{\bottomfraction}{.9}
\renewcommand{\textfraction}{.1}
\renewcommand{\familydefault}{\sfdefault} %setting the san serif font

%%%%%%%%%%%%%%%%%%%%%%%% Line spacing
% Use the following command for ``double'' spacing
%\setlength{\baselineskip}{1.2\baselineskip}
% and this one for an acceptable NIH spacing of 6lpi based on 11pt
%\setlength{\baselineskip}{.9\baselineskip}
% The baselineskip does not appear to work when we include a maketitle
% command in the main file.  Something there must set the line spacing
% If we use this next command, then things seem to work.
\renewcommand{\baselinestretch}{.9}

\setcounter{secnumdepth}{0} %make no numbers but have a table of contents


\begin{document}

\title{Lab 4, Neuron and Backyard Brains}
\author{Jake Bergquist, u6010393}
\maketitle
\tableofcontents
\newpage

\section{Introduction}
In biology the study of structure and function relationships relies in part on various imaging techniques in order to assess the physical structure and layout of biological materials. At an organ and even to some extend a tissue level methods such as visible light microscopy cna be beneficial for resolving anatomical and cellular structure. However, when it comes to assessing biological specimens at a cellular and sub cellular level, higher resolution and finer detail are often required. Many of these methods of imaging of cellular level structures and organization rely on fluorescent labeling. This is a method by which specific substructures of a cell are tagged with material can be identified using florescent microscopy. The materials used to tag the cells are typically proteins conjugated to florescent molecules (in many cases a primary antibody protein attaches to the structure of interest and secondary antibody attaches to the primary antibody and carries the fluorescent molecule) or in some cases the fluorescent molecules themselves can act as specific or general tags. The targets of these tags range from specific proteins, protein subdomains, amino acid sequences, DNA, and other cellular components. Tissue samples are fixed and tagged using these various systems with several different fluorescent tags used simultaneously to assess multiple structures of interest within one tissue and there localization relative to one another. These fluorescent tags are visualized using fluorescent microscopy. This is a process by which the fluorescent molecules are induced to emit a wavelength of light which is specific to each fluorescent molecule. Each fluorescent molecule is characterized by an excitation and emission spectrum, which describes the wavelengths of light that will excite the electrons of the florescent molecule (the excitation spectrum), and what wavelength of light is released when the electrons fall back to their basal energy state (the emission spectrum). By shining light onto the specimen int he excitation spectrum and recording light from the emission spectrum the specific fluorescent molecules can be imaged. By adjusting the focus of the light shined onto and collected from the specimen down to a single point and scanning this point across the specimen and through its depth the 2D and 3D structure of the specimen can be resolved.

The specific tissue of interest in this report was taken from human atria. These tissue samples were labeled with wheat  germ  agglutinin (WGA)  conjugated  to  AlexaFluor-488 which stains the extracellular space and a stain for connexin43 (Cx43) with AlexaFluor-594. This allowed us to separately resolve the extracellular space which defines the spae of the cardiac myocytes as well as the spatial arrangement of the gap junctions (which contain two hexamers of connexin43/other connexins). Cardiac myocytes are typically cylindrical or ``brick" shaped cells that arrange into filaments and sheets.\cite{Woodcock2005} The gap junctions composed of two hexamers of connexins form open channels between cardiac myocytes. Gap junctions are found across the cell membrane where the myocytes is connected with its neighbor but the concentration is higher at the terminal ends of the myocytes.\cite{Mese2007} These gap junctions allow for free passage of ions and small molecules from cell to cell. This allows for the electrical activity of one cell such as the initiation of an action potential, to propagate to the neighboring cells. In this way the activating wavefront of cardiac electrical activity propagates from cell to cell in the heart. This leads to a synchronized and ordered wave of electrical activation across the heart which in turn leads to an ordered contraction of the heart.\cite{Mese2007} The arrangement of gap junctions throughout the cell plays a critical role in controlling the direction and propagation properties of this excitation wave. As such, visualizing the structure and arrangement of gap junctions in myocytes can be informative when examining disease states where gap junction remodeling is implicated.

\section{Methods}
\subsection{Image Acquisition}
Atrial tissue samples were obtained fromt he lab instructor. These samples had be prepared with WGA conjugated to AlexaFluor-488 stain and a connexin43 stain conjugated to AlexxaFluor-594. The sample 
%%%%%%%%%%%%%%%%%% Correct Bibliography Style

\bibliography{C:/Users/Jake/Documents/library}
\bibliographystyle{ieeetr}


\end{document}








