\documentclass[12pt]{article}
\usepackage{graphicx}
\usepackage {color}
\usepackage{pdfpages}
\usepackage{float}
\usepackage{changebar}
\usepackage{enumitem,amssymb}
\renewcommand{\familydefault}{\sfdefault}
\usepackage[margin=1.2in]{geometry}
\usepackage{graphicx}
\usepackage{wrapfig}
\usepackage[super]{cite}
\usepackage{subcaption}
\usepackage[table]{xcolor}
\usepackage{amsmath}
\usepackage[sort, numbers]{natbib}

%%%%%%%%%%%%Defining the margins %%%%%%%%%%%%%%%%%%%%%
\textheight 9.in
\textwidth 6.5in
\topmargin -.5in
\oddsidemargin 0in
\setlength{\parskip}{\smallskipamount}

%%%%%%%%%%%%%%Specific Commands %%%%%%%%%%%%%%%%%%
\newcommand{\eg}{{\em e.g.,}}
\newcommand{\ie}{{\em i.e.,}}
\newcommand{\etc}{{\em etc.,}}
\newcommand{\etal}{{\em et al.}}
\newcommand{\degrees}{{$^{\circ}$}}
\newcommand{\fig}[1]{\textbf{Figure #1}}

%%%%%%%%%%%%%%%%%%%%%%%%%%%% Setting to control figure placement
% These determine the rules used to place floating objects like figures 
% They are only guides, but read the manual to see the effect of each.
\renewcommand{\topfraction}{.9}
\renewcommand{\bottomfraction}{.9}
\renewcommand{\textfraction}{.1}
\renewcommand{\familydefault}{\sfdefault} %setting the san serif font

%%%%%%%%%%%%%%%%%%%%%%%% Line spacing
% Use the following command for ``double'' spacing
%\setlength{\baselineskip}{1.2\baselineskip}
% and this one for an acceptable NIH spacing of 6lpi based on 11pt
%\setlength{\baselineskip}{.9\baselineskip}
% The baselineskip does not appear to work when we include a maketitle
% command in the main file.  Something there must set the line spacing
% If we use this next command, then things seem to work.
\renewcommand{\baselinestretch}{.9}

\setcounter{secnumdepth}{0} %make no numbers but have a table of contents


\begin{document}

\title{Coronary Artery Disease Mediated Myocardial Ischemia and its Effects on the Body }
\author{Jake Bergquist, u6010393}
\maketitle

\section{Introduction}

-Coronary Artery Disease affects (large number) annually

-At currently defined clinically significant levels it causes myocardial ischemia, and eventual infarction

-Sub clinical levels of disease can present as ischemia seen in routine exercise stress tests, but not always

-theoretically during such stress tests ischemia is present, but it is not always detectable

-Stress tests are done either mechanically (running on a treadmill) or chemically (a stimulant), and they are treated the same, but the ischemia that develops might be different

-An understanding of how ischemia develops under conditions of subclinical coronary artery disease during cardiac stress tests would improve the understanding of the effects on the body and improve our ability to detect it during such stress tests in order to diagnose pathologies such as sub clinical coronary artery disease

-computational Modeling approaches for cardiac ischemia have developed quickly in the last decade and can be used to improve our understanding of how various ischemia patterns present

-By combining the modeling approaches 

\section{Background}

\section{Methods}

\section{Results}

\section{Discussion}

%%%%%%%%%%%%%%%%%% Correct Bibliography Style

\bibliography{C:/Users/Jake/Documents/library}
\bibliographystyle{IEEEtran}
*Note a non English citation, but an English translation of the text was obtained and used

\end{document}








