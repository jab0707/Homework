\documentclass[12pt]{article}
\usepackage{graphicx}
\usepackage {color}
\usepackage{pdfpages}
\usepackage{float}
\usepackage{changebar}
\usepackage{enumitem,amssymb}
\renewcommand{\familydefault}{\sfdefault}
\usepackage[margin=1.2in]{geometry}
\usepackage{graphicx}
\usepackage{wrapfig}
\usepackage[super]{cite}
\usepackage{subcaption}
\usepackage[table]{xcolor}
\usepackage{amsmath}
\usepackage[sort, numbers]{natbib}

%%%%%%%%%%%%Defining the margins %%%%%%%%%%%%%%%%%%%%%
\textheight 9.in
\textwidth 6.5in
\topmargin -.5in
\oddsidemargin 0in
\setlength{\parskip}{\smallskipamount}

%%%%%%%%%%%%%%Specific Commands %%%%%%%%%%%%%%%%%%
\newcommand{\eg}{{\em e.g.,}}
\newcommand{\ie}{{\em i.e.,}}
\newcommand{\etc}{{\em etc.,}}
\newcommand{\etal}{{\em et al.}}
\newcommand{\degrees}{{$^{\circ}$}}
\newcommand{\fig}[1]{\textbf{Figure #1}}

%%%%%%%%%%%%%%%%%%%%%%%%%%%% Setting to control figure placement
% These determine the rules used to place floating objects like figures 
% They are only guides, but read the manual to see the effect of each.
\renewcommand{\topfraction}{.9}
\renewcommand{\bottomfraction}{.9}
\renewcommand{\textfraction}{.1}
\renewcommand{\familydefault}{\sfdefault} %setting the san serif font

%%%%%%%%%%%%%%%%%%%%%%%% Line spacing
% Use the following command for ``double'' spacing
%\setlength{\baselineskip}{1.2\baselineskip}
% and this one for an acceptable NIH spacing of 6lpi based on 11pt
%\setlength{\baselineskip}{.9\baselineskip}
% The baselineskip does not appear to work when we include a maketitle
% command in the main file.  Something there must set the line spacing
% If we use this next command, then things seem to work.
\renewcommand{\baselinestretch}{.9}

\setcounter{secnumdepth}{0} %make no numbers but have a table of contents


\begin{document}

\title{Differences in Chemical vs Exercise Induced Ischemia: Implications for Development of Ischemic Models }

\author{Jake Bergquist, u6010393}
\maketitle

\section{Introduction}

-Coronary Artery Disease affects (large number) annually

-At currently defined clinically significant levels it causes myocardial ischemia, and eventual infarction

-Sub clinical levels of disease can present as ischemia seen in routine exercise stress tests, but not always

-theoretically during such stress tests ischemia is present, but it is not always detectable

-Stress tests are done either mechanically (running on a treadmill) or chemically (a stimulant), and they are treated the same, but it has been shown that there are differences in the physiological response.\cite{Beleslin1994} Additionally this work shows differences in sensitivity and specificity of these methods.

-An understanding of how ischemia develops under conditions of subclinical coronary artery disease during cardiac stress tests would improve the understanding of the effects on the body and improve our ability to detect it during such stress tests in order to diagnose pathologies such as sub clinical coronary artery disease

-computational Modeling approaches for cardiac ischemia have developed quickly in the last decade and can be used to improve our understanding of how various ischemia patterns present electrically

-By combining the modeling approaches with our understanding of how these different types of stress tests induce different ischemia we can better formulate models for understanding and diagnosing such ischemia

\section{Background}

-Introduction to the structure and function of coronary vasculature

-Discuss aberrations present in coronary vascular disease (plaques, hardening) and how these affect cardiac perfusion 

-Introduce how clinical stress tests are conducted and the difference between chemical and physical stress tests

-Segment about ecgi in reference to stress tests

-Introduce the concepts of computational modeling of ischemia on a whole heart scale


\section{Methods}
Paper 1:\cite{Beleslin1994}
-Human patients identified to have significant coronary artery disease and accompanying ischemia via invasive methods were subjected to ischemic stress tests. Physiological differences between Bruce and chemical stress tests as well as a difference in specificity and sensitivity and overall accuracy were assessed. 

-Limitations: No in depth analysis of the electrical signals from these patients. Analysis of the ischemia created was limited. Was performed in human subjects therefore no in depth dissection or measurement of the ischemia at the source could be performed (ie with direct recording electrodes). Study did not attempt to explain differences in response to different stress test stressors.



Paper 2:\cite{Burton2018}

-Formulates and ischemic model based on in situ large animal induced ischemic events where ischemia is monitored by intramural and epicardial electrical recordings.

-Utilizes a passive bidomain formulation to solve for the extracellular potentials throughout the heart given prescribed transmembrane voltages within the measured ischemic zone utilizing a finite element solution. 

-Limitations: Utilizes data from one kind of ischemic stress not representative of an actual stress test. Border zone surrounding the ischemic zone is characterized by a uniform dual exponential decay of trans membrane potential. Comparison to measured epicardial potentials shows partial but not complete matching.


\section{Results}
Paper 1:
-The three different stress tests used (exercise, dobutamine, and dipyridamole) showed variation in the physiological parameters measured as well as a difference in sensitivity and specificity to detecting ischemia.

-paper concludes that despite these differences chemical stressors can serve in place of exercise stress tests in patients who are unwilling/unable to perform the standard Bruce exercise stress test.

Paper 2:

-was able to recapitulate ischemic extracellular potentials on the epicardium using modeled ischemic transmembrane sources. Demonstrated that the passive bidomain can be used to model ischemia within the heart.


\section{Discussion}

-While paper 1 concludes that the stress tests examined are all adequate for detecting ischemia, there are clear differences in the pysiological response. This implies a difference in the ischemia induced. This idea of different ischemia produced by different stress tests is supported by other studies.\cite{Zenger2019}

-Paper 1 highlights the need to further investigate these differences. By applying the method of paper 2 we may gain a deeper understanding of the ischemia that developed. As a consequence of this proposed integration the findings may inform clinicians on how better to interpret the results of the different stress tests, leading to more accurate and effective diagnosis of cardiac ischemia.

%%%%%%%%%%%%%%%%%% Correct Bibliography Style

\bibliography{C:/Users/Jake/Documents/library}
\bibliographystyle{IEEEtran}


\end{document}








