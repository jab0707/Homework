\documentclass[12pt]{article}
\usepackage{graphicx}
\usepackage {color}
\usepackage{pdfpages}
\usepackage{float}
\usepackage{changebar}
\usepackage{enumitem,amssymb}
\renewcommand{\familydefault}{\sfdefault}
\usepackage[margin=1.2in]{geometry}
\usepackage{graphicx}
\usepackage{wrapfig}
\usepackage[super]{cite}
\usepackage{subcaption}
\usepackage[table]{xcolor}
\usepackage{amsmath}
\usepackage[sort, numbers]{natbib}
\usepackage{multirow}
\usepackage{tabularx}
\usepackage{siunitx}

%%%%%%%%%%%%Defining the margins %%%%%%%%%%%%%%%%%%%%%
\textheight 9.in
\textwidth 6.5in
\topmargin -.5in
\oddsidemargin 0in
\setlength{\parskip}{\smallskipamount}

%%%%%%%%%%%%%%Specific Commands %%%%%%%%%%%%%%%%%%
\newcommand{\eg}{{\em e.g.,}}
\newcommand{\ie}{{\em i.e.,}}
\newcommand{\etc}{{\em etc.,}}
\newcommand{\etal}{{\em et al.}}
\newcommand{\degrees}{{$^{\circ}$}}
\newcommand{\fig}[1]{\textbf{Figure #1}}
\DeclareMathOperator*{\argmin}{argmin}
%%%%%%%%%%%%%%%%%%%%%%%%%%%% Setting to control figure placement
% These determine the rules used to place floating objects like figures 
% They are only guides, but read the manual to see the effect of each.
\renewcommand{\topfraction}{.9}
\renewcommand{\bottomfraction}{.9}
\renewcommand{\textfraction}{.1}
\renewcommand{\familydefault}{\sfdefault} %setting the san serif font

%%%%%%%%%%%%%%%%%%%%%%%% Line spacing
% Use the following command for ``double'' spacing
%\setlength{\baselineskip}{1.2\baselineskip}
% and this one for an acceptable NIH spacing of 6lpi based on 11pt
%\setlength{\baselineskip}{.9\baselineskip}
% The baselineskip does not appear to work when we include a maketitle
% command in the main file.  Something there must set the line spacing
% If we use this next command, then things seem to work.
\renewcommand{\baselinestretch}{.9}
\newcommand{\rpm}{\raisebox{.2ex}{$\scriptstyle\pm$}}
\setcounter{secnumdepth}{0} %make no numbers but have a table of contents


\begin{document}

\title{Term Project: Optimization of Reconstruction}
\author{Jake Bergquist, u6010393 }
\maketitle

\section{Introduction}
Work done in our labs has shown that an incomplete sampling of the epicardial
potentials during our typical experimental preparations creates significant error
during computational modeling of the heart.\cite{RSM:Tat2018b} Many experimental preparations such as those used in our labs as well as those of collaborators utilize these epicardial sock electrode arrays that only provide partial coverage of the cardiac source.\cite{RSM:Bur2018b,Zenger2019,RSM:Goo2018,RSM:Bea2015a} In such
cases, we must aproximate the missing data on the atria from the signals we do recover on the ventricles. A
common approach that we have used in the past is to approximate the missing signals by interpolating from the
measured data often using a surface Laplacian approach, however, it is
unclear if interpolation approaches are able to reconstruct the missing
atrial signals in a scenario in which only ventricular signals are
measured.\cite{Huiskamp1991}

To address the difficulties we encounter with this reconstruction we developed an experimental setup where we utilize a rigid peri-cardiac cage that forms a closed surface of recording electrodes around the cardiac source. This cage is constructed such that each electrode has equal solid angle to the center of the cage and should capture the entire cardiac surface.

Before addressing the reconstruction techniques I need to employ on the epicardial sock electrode array I wanted to work with a more simple case in which only the cardiac cage was considered. I divided the cage into a ventricular (lower) segment and an segment. I then formulated two optimization problems in order to address the reconstruction of the upper atrial portion of the cage using the lower ventricular signals. The key difference between the two methods was one utilized only the time signals while the other optimized a term for the time derivative of the signals. I tested these methods across three beat morphologies. I found good agreement of the reconstructed signals when compared to the measured. By contrast the Laplacian interpolation over the upper region performed poorer in all cases, specifically with more complicated beat morphologies resulting int he poorest Laplacian reconstruction.

I then formulated an optimization of the reconstruction of the atrial signals from the ventricular signals by informing the reconstruction using the cage potentials. Utilizing the boundary element formulation I can solve laplaces equation through the volume between the epicardial sock and the cage to calculate what the potential distribution should be on the cage given a certain sock potential distribution. This formulation however requires a closed inner source surface, hence why we need to recreate the atrial potentials. By minimizing the difference between these forward computed potentials and the measured cage potentials over an interpolation/reconstruction matrix that gives us our atrial signals we can optimize this reconstruction matrix.

\section{Methods}

The pericardiac cage utilized is shown in Figure~\ref{Fig:Cag}. As can be seen the cage is made up of many (256) recording electrode positions held in place by a 3D printed frame. The potential distribution (in milivolts) at a specific time instance during a heartbeat is shown the left in Figure~\ref{Fig:Cag} demonstrating that the cage is able to record simultaneous electrical recordings from the entire cage surface. This cage was divided in a manner similar to the scenario seen in experimental situations where the atrial region (top) is considered missing data that the ventricular (bottom) region must be used to reconstruct. The atrial region ($C_A$) is comprised of the top quarter of the cage (the top 4 rings) and the ventricular region ($C_V$) is comprised of the remaining three quarters of the cage. I then designed a basic optimization approach to calculate a matrix, $D$ that maps $C_V$ to $C_A$. By minimizing Equation~\ref{EQ:Opt1} over $D$ we obtain an optimal $D$.

\begin{figure}[h]
	% \centering
	\includegraphics[width=0.9\textwidth]{Figures/CageFigure.png}
	\caption{The Utah Pericardiac Cage (UPC)[. The UPC] is a 3D printed
		structure that has mounting holes for 256 recording
		electrodes; one such electrode location is circled in red. The electrode locations connect to form
		triangles, all of which have approximately equal solid
		angles with respect to the centroid of the cage. The two halves of
		the UPC are closed around the isolated heart and both the UPC and
		heart are immersed in a torso tank filled with
		electrolyte. The recorded signals can be mapped onto a digitized
		geometry of the electrode locations. Colors here represent the
		potential distribution at a time instant 50\% into the QRS of a sinus
		beat.}
	\label{Fig:Cag}
\end{figure}



\begin{equation}
\argmin_{D} ||DC_V - C_A||^2
\label{EQ:Opt1}
\end{equation}

\begin{equation}
(DC_V - C_A)^T(DC_V - C_A)
\label{EQ:Opt1_1}
\end{equation}

\begin{equation}
C_V^TD^TDC_V  - 2C_V^TD^TC_A - C_A^TC_A
\label{EQ:Opt1_2}
\end{equation}

\begin{equation}
D(2C_VC_V^T) - 2C_AC_V^T - 0 = 0
\label{EQ:Opt1_3}
\end{equation}

\begin{equation}
D = C_AC_V^T(C_VC_V^T)^{-1}
\label{EQ:Opt1_4}
\end{equation}

\begin{equation}
D^{I+1} = D^I - \epsilon(2D^NC_VC_V^T - 2C_AC_V^T)
\label{EQ:Opt1_5}
\end{equation}

We can solve this optimization problem by expanding Equation~\ref{EQ:Opt1} into Equation~\ref{EQ:Opt1_1} which can be epanded further to give Equation~\ref{EQ:Opt1_2}. Taking the derivative with respect to $D$ and setting this equal to zero gives Equation~\ref{EQ:Opt1_3}. From here we can either solve using gradient descent (Equation~\ref{EQ:Opt1_5}) or by simply solving for the optimal $D$ as in Equation~\ref{EQ:Opt1_4}. The second optimization formulation takes into account an additional parameter. Because $C_V$ and $C_A$ are comprised of time signals it would make sense to include information about the time derivative of these signals in the form of another interpolation matrix that weights these time derivative terms. We see this in Equation~\ref{EQ:Opt2} where not only do we weight the time signals ($E$) but also we add a mapping matrix, $F$, that weights the derivative of $C_V$ with respect to time ($C'_v$).

\begin{equation}
\argmin_{E,F} ||EC_V + FC'_V - C_A||^2
\label{EQ:Opt2}
\end{equation}

\begin{equation}
C_V^TE^TEC_V + 2C_V^TE^TFC'_V - 2C_V^TE^TC_A + C'_V^TF^TFC'_V - 2C'_V^TF^TC_A + C_A^TC_A
\label{EQ:Opt2_1}
\end{equation}

\begin{equation}
2EC_VC_V^T + 2FC'_VC_V^T - 2C_AC_V^T +0 +0 +0
\label{EQ:Opt2_2}
\end{equation}

\begin{equation}
E^{i+1} = E^i - \epsilon_1(2(E^iC_V + FC'_V - C_A)C_V^T)
\label{EQ:Opt2_3}
\end{equation}

\begin{equation}
2FC'_VC'_V^T + 2EC_VC'_V^T - 2C_AC'_V^T +0 +0 +0
\label{EQ:Opt2_4}
\end{equation}

\begin{equation}
E^{i+1} = E^i - \epsilon_2(2(F^iC'_V + EC_V - C_A)C'_V^T)
\label{EQ:Opt2_5}
\end{equation}

By expanding the formulation in Equation~\ref{EQ:Opt2} we get Equation~\ref{EQ:Opt2_1}. Because we have two variables to optimize ($E$ and $F$) we first take the derivative of Equation~\ref{EQ:Opt2_1} with respect to $E$ which yields Equation~\ref{EQ:Opt2_2}. We can use this to calculate the gradient step of $E$ as Equation~\ref{EQ:Opt2_3}. By repeating this process with respect to $F$ we get Equations~\ref{EQ:Opt2_4}~and~\ref{EQ:Opt2_5}. By taking simultaneous steps in $E$ and $F$ both can be optimized.

As a comparison to these methods I computed a surface Laplacian interpolation using the SCIRun problem solving environment in order to interpolate the $C_V$ potentials onto the atrial region. The resulting three reconstructions were calculated and compared using a premature ventricular contraction, a sinus beat, and a paced sinus beat morphology. The resulting maximum and mean error in the reconstructed potential was calculated. For visualization a time point during the QRS complex, the most dynamic section of the heartbeat, was visualized on a planar projection of the reconstructed atrial region. A planar projection was performed to allow for the simultaneous viewing of the reconstructed area despite its curvature.

\section{Results}

The numerical results summaraized in Table~\ref{tab:err} show that for the reconstruction of the atrial portion of the cage the optimization methods performed similarly well, while the Laplacian interpolation performed poorly for both the sinus and atrially paced beats. The paced ventricular beat (premature ventricular contraction) beat morphology is considerably simpler than a sinus beat and in this case the Laplacian interpolation performed comparably to the other optimization based methods.

\begin{table}[htbp]
	\centering
	\caption{\label{tab:err} Numerical results from the three reconstruction scenarios. Maximum, mean, and standard deviation or error displayed in mV.}
	%\centering
	\begin{tabular}{|lr|r|r|r|} \hline
		&&Ventricular&Sinus&Atrial\\
		Method&&Paced&&Paced\\
		&&(mV)&(mV)&(mV)\\
		\hline
		\multirow{3}{*}{Opt. 1}
		&Max Err.&0.35&0.16&1.0\\
		&Mean Err.&0.011&0.011&0.017\\
		&STD&\rpm{}0.018&\rpm{}0.013&\rpm{}0.039\\
		\hline
		\multirow{3}{*}{Opt. 2}
		&Max Err.&0.41&0.21&1.0\\
		&Mean Err.&0.013&0.015&0.017\\
		&STD&\rpm{}0.021&\rpm{}0.016&\rpm{}0.040\\
		\hline
		\multirow{3}{*}{Laplacian}
		&Max Err.&1.5&2.6&11\\
		&Mean Err.&0.071&0.11&0.10\\
		&STD&\rpm{}0.11&\rpm{}0.18&\rpm{}0.38\\
		\hline
		
	\end{tabular}
\end{table}

The potential distributions of the Laplacian interpolated and the optimized
reconstructed signals varied greatly. The two optimized reconstructions
resulted in similar distributions that agreed with the measured
distributions. The Laplacian interpolated signals showed reasonable
agreement with the measured signals for the ventricular paced beat
morphologies. However, there was striking disagreement between the
Laplacian interpolated and measured signals for both the sinus and atrial
paced morphologies. The potential distributions for each beat
at 50\% into the QRS complex are displayed in Figure~\ref{Fig:Res}. Only one of the optimization method reconstructions is shown as the optimization method potential distributions are indistinguishable at this time point.


\begin{figure}[h]
	% \centering
	\includegraphics[width=0.9\textwidth]{Figures/Figure1.png}
	\caption{Potentials on the atrial subset of the cage for each
		reconstruction method across three beat morphologies. The time point
		displayed for each case is 50\% into the QRS complex. The measured
		column shows the potential distribution measured from the cage. The optimized
		reconstruction column shows potentials from the Opt. 1 method. The
		Laplacian Interpolation column shows the potentials from the spatial
		Laplacian interpolation. Row A shows a ventricular paced beat. Row B
		shows sinus beat. Row C shows an atrial paced beat.}
	\label{Fig:Res}
\end{figure}

When comparing the signals directly we see in Figure~\ref{Fig:Sigs_cg} when we select an electrode on the top of the cage in the $C_A$ region we see that for the ventricular paced beat all three of the reconstruction methods (Opt. 1, Opt. 2, and Laplacian interpolation) perform similarly well in reconstructing the measured signal. However when we consider a beat with a more complex signal morphology such as an atrial paced beat we see that while the two optimization based methods perform well, the Lalacian interpolation performs poorer. An investigation of the weights calculated by the optimization methods is shown in Figure~\ref{Fig:reco_vis}. As can be seen, both the linear components of the optimization methods ($D$ for Opt. 1, and $E$ for Opt 2) select weights near the top of the border between the cage near the electrode that is being reconstructed with some pattern of weights on the remaining cage. However, the time derivative weights in Opt. 2 ($F$) also weight a region of the cage associated with the activation front, as seen by the potential distribution.

\begin{figure}[h]
	% \centering
	\includegraphics[width=0.9\textwidth]{Figures/cgSigs.png}
	\caption{Time signals taken at one location on the atrial portion of the cage ($C_A$) for each of the reconstruction methods. The measured signal (A), Optimization method 1 reconstruction (B), Optimization method 2 (B), and Laplacian interpolation (D). The left column shows a ventricular paced beat morphology, and the right column shows an atrial paced beat morphology.}
	\label{Fig:Sigs_cg}
\end{figure}
\begin{figure}[h]
	\centering
	\includegraphics[width=0.9\textwidth]{Figures/reco_vis.png}
	\caption{Patterns of the weights of the reconstruction matrices as compared to a potential distribution during the QRS complex. The potential distribution shown as $C_V$ and $C_A$ is taken during the QRS complex to highlight the morphology of the activation front. The specific electrode weights for $D$, $E$, and$F$ is one on same side of the cage as the breakthrough site. The electrode being reconstructed is highlighted with the purple circle. Units for $C_V$ and $C_A$ in mV, all weights are unit-less, $D$ and $F$ share a color bar.}
	\label{Fig:reco_vis}
\end{figure}


\section{Discussion}

The main goal of this project was to investigate utilization of reconstruction methods for reconstructing of sampling of one part of the cardiac source from another part across different beat morphologies. I then compared the optimization based methods to the standard Laplacian interpolation which is common in this field.  I found that not only were we able to reconstruct the entire recorded signals from only a subset with high fidelity but also that the optimization based methods greatly outperformed Laplacian interpolation. This study leveraged an ideal scenario using the optimization problem in
which the desired missing data was available and could
be incorporated into the formulation, which is often not the case in
typical experimental preparations. This investigation showed that missing signals
during experiments can be reconstructed from measured signals with high
fidelity if the reconstruction is formulated properly. Thus, the results
from the optimization approaches provide a benchmark of ``ideal
reconstruction'' against which to compare other methods such as
Laplacian interpolation.  Such optimization approaches could also be
reformulated to incorporate inverse computed potentials or other relevant
measurements in place of the of the missing data. The optimization could
also be reformulated to minimize the difference between the fully
reconstructed and measured potentials forward computed to another
measurement surface (such as a torso surface) and the measured potentials
at that other surface, as a way to optimize the reconstruction mapping
matrix. Once computed, the reconstruction mapping matrices ($D$, or $E$ and
$F$) could be then used to reconstruct the missing potentials in
subsequent beats of a similar morphology. I did find that utilizing the optimized mapping matrices  created by utilizing one beat morphology did not translate well to other beat morphologies. This suggests that the optimization formulations are not taking into account some form of non linear behavior. Further investigation into these non linearities could elucidate their relationships and lead to a more robust formulation. 

In an attempt to identify what the optimization methods were identifying to use for reconstruction I visualized the weights from the different mapping matrices to try to identify patterns that could be generalized or at least under to understand what is missing from conventional reconstruction approaches. In Figure~\ref{Fig:reco_vis} we see that the linear portions fo the reconstructions, either $D$ or $E$ behave mostly as expected, by weighting signals highly that are close to the electrode of interest (circled in purple). However what is odd is the apparent negative weighting in a strange pattern across the remainder of the ventricular portion of the cage as well as a very high weighting at the apex of the cage. While the weights close to the electrode of interest mimic what a Laplacian interpolation would do, the remaining pattern is unexpected and difficult to interpret. When I look at the weights of the derivative of the signals with respect to time something immediately jumps out at me. The weights pattern for electrodes near the breakthrough site (the site on the cage that sees the QRS complex the earliest) mimics shape of the activation front/breakthrough site to some degree. This trend dissipates when the electrode of interest is farther from this site, but the presence of a pattern like this is interesting to find. I see a similar pattern for all beat morphologies I investigated, but I chose to show it using a paced ventricular beat as that morphology is the simplest.

This project implies that there is plenty of improvement yet to be made on these reconstruction problems in the context of electrophysiology, and that the standard approaches such as Laplacian interpolation, may not be sufficient in some cases. Other work by our collaborations has shown that Laplacian interpolation performs well when the samples being used for interpolation surround the missing area, however int he case presented by this project perhaps interpolation is a misnomer. Given the geometry of the cage it is more like extrapolation to the upper region of the cage rather than interpolation. While there are several contexts where interpolation is needed, such as interpolating on body surface electrode recordings to allow for the use of limited lead sets, there are other scenarios such as those encountered in our lab where a more extrapolation like problem is the case. I plan to utilize the techniques developed in this project to refine our understanding of such extrapolation and interpolation problems in order to formulate more accurate and robust methods. In particular I have begun working on a similar approach as the one presented here to reconstruct the potentials on the atrial region of the epicardial surface given only ventricular sock recordings. My hope is that this work will better inform our computational models with the eventual end goal being to improve the formulation and use of the electrocardiography inverse problem.


\section{Acknowledgments}
I would like to thank my fellow graduate students who collected this data with me, and helped me think about these problems. My adviser who helped direct my study, my funding sources for paying me, and you Dr. Joshi, for working with me through my limited mathematical background to get me to a point where I can perform interesting studies such as this one with techniques I have learned in your class.

%%%%%%%%%%%%%%%%%% Correct Bibliography Style

\bibliography{library,biglit,strings}
\bibliographystyle{IEEEtran}


\end{document}








