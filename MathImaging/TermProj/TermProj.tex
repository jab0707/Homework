\documentclass[12pt]{article}
\usepackage{graphicx}
\usepackage {color}
\usepackage{pdfpages}
\usepackage{float}
\usepackage{changebar}
\usepackage{enumitem,amssymb}
\renewcommand{\familydefault}{\sfdefault}
\usepackage[margin=1.2in]{geometry}
\usepackage{graphicx}
\usepackage{wrapfig}
\usepackage[super]{cite}
\usepackage{subcaption}
\usepackage[table]{xcolor}
\usepackage{amsmath}
\usepackage[sort, numbers]{natbib}
\usepackage{multirow}
\usepackage{tabularx}
\usepackage{siunitx}

%%%%%%%%%%%%Defining the margins %%%%%%%%%%%%%%%%%%%%%
\textheight 9.in
\textwidth 6.5in
\topmargin -.5in
\oddsidemargin 0in
\setlength{\parskip}{\smallskipamount}

%%%%%%%%%%%%%%Specific Commands %%%%%%%%%%%%%%%%%%
\newcommand{\eg}{{\em e.g.,}}
\newcommand{\ie}{{\em i.e.,}}
\newcommand{\etc}{{\em etc.,}}
\newcommand{\etal}{{\em et al.}}
\newcommand{\degrees}{{$^{\circ}$}}
\newcommand{\fig}[1]{\textbf{Figure #1}}

%%%%%%%%%%%%%%%%%%%%%%%%%%%% Setting to control figure placement
% These determine the rules used to place floating objects like figures 
% They are only guides, but read the manual to see the effect of each.
\renewcommand{\topfraction}{.9}
\renewcommand{\bottomfraction}{.9}
\renewcommand{\textfraction}{.1}
\renewcommand{\familydefault}{\sfdefault} %setting the san serif font

%%%%%%%%%%%%%%%%%%%%%%%% Line spacing
% Use the following command for ``double'' spacing
%\setlength{\baselineskip}{1.2\baselineskip}
% and this one for an acceptable NIH spacing of 6lpi based on 11pt
%\setlength{\baselineskip}{.9\baselineskip}
% The baselineskip does not appear to work when we include a maketitle
% command in the main file.  Something there must set the line spacing
% If we use this next command, then things seem to work.
\renewcommand{\baselinestretch}{.9}

\setcounter{secnumdepth}{0} %make no numbers but have a table of contents


\begin{document}

\title{Term Project: Optimization of Reconstruction}
\author{Jake Bergquist, u6010393 }
\maketitle

\section{Introduction}
Work done in our labs has shown that an incomplete sampling of the epicardial
potentials during our typical experimental preparations creates significant error
during computational modeling of the heart.\cite{RSM:Tat2018b} Many experimental preparations such as those used in our labs as well as those of collaborators utilize these epicardial sock electrode arrays that only provide partial coverage of the cardiac source.\cite{RSM:Bur2018b,Zenger2019,RSM:Goo2018,RSM:Bea2015a} In such
cases, we must aproximate the missing data on the atria from the signals we do recover on the ventricles. A
common approach that we have used in the past is to approximate the missing signals by interpolating from the
measured data often using a surface Laplacian approach, however, it is
unclear if interpolation approaches are able to reconstruct the missing
atrial signals in a scenario in which only ventricular signals are
measured.\cite{Huiskamp1991}

To address the difficulties we encounter with this reconstruction we developed an experimental setup where we utilize a rigid peri-cardiac cage that forms a closed surface of recording electrodes around the cardiac source. This cage is constructed such that each electrode has equal solid angle to the center of the cage and should capture the entire cardiac surface.

Before addressing the reconstruction techniques we need to employ on  the epicardial sock electrode array we wanted to work with a more simple case in which only the cardiac cage was considered. 

\section{Methods}


\section{Results}




\section{Discussion}

%%%%%%%%%%%%%%%%%% Correct Bibliography Style

\bibliography{library,biglit}
\bibliographystyle{IEEEtran}


\end{document}








